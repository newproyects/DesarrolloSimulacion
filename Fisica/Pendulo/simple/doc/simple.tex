\documentclass[10pt,a4papper]{article}
\usepackage{graphicx}
\usepackage{amsmath}
\usepackage{amssymb}
\usepackage{cancel}
\usepackage[hidelinks]{hyperref}
\usepackage[left=1.00cm, right=3.00cm, top=2.00cm, bottom=2.00cm]{geometry}
\author{Angel Fdo. García Núñez}
\date{Septiembre 17, 2020}
\title{Estadisitica}

\begin{document}

\Huge
Desarrollo y simulación de péndulo simple\\

Angel Fernando García Núñez

\newpage
\LARGE
Formulación de Newton\\\\
\Large
\[\sum\vec F=m\vec a\]\\
Sumatoria de Fuerzas
\[\vec W\equiv\text{peso}\]
\[\vec T\equiv\text{tensión}\]\\
\[\sum\vec F=\vec T+\vec W\]\\
Descomposición de fuerzas(eje normal y eje tangente)
\[\sum\vec F=\sum\vec F_n+\sum\vec F_t\quad,\quad\sum F=\left|\sum\vec F\right|\]
\[\vec W=\vec W_n+\vec W_t\quad,\quad W=|\vec W|\to W=mg\]
\[\vec T=|\vec T|\hat n\quad,\quad T=|\vec T|\to\vec T=T\hat n\]\\
Fuerzas en el Eje normal
\[\vec W_n=|\vec W|\vec n\to\vec W_n=-W\cos\theta\hat n\]
\[\vec W_n=-mg\cos\theta\hat n\]\\
\[\sum\vec F_n=0\to\vec T+\vec W_n=0\]
\[\vec T=mg\cos\theta\hat n\to T=mg\cos\theta\]\\
Fuerzas en el Eje tangente
\[\vec W_t=|\vec W|\vec t\to\vec W_t=-W\sin\theta\hat t\]
\[\quad\sum\vec F_t=\vec W_t\to\sum\vec F_t=-mg\sin\theta\hat t\to\sum F_t=-mg\sin\theta\]

\newpage
Ecuación de movimiento
\[\sum\vec F=\cancelto{0}{\sum\vec F_n}+\sum\vec F_t\to\sum\vec F=\sum\vec F_t\]
\[\sum\vec F_t=m\vec a_t\]
\[m\vec a_t=-mg\sin\theta\hat t\to a_t=-g\sin\theta\]\\
\[\vec a_t=\vec r\times\vec\omega\quad,\quad|\vec r|=l\equiv\text{cte.}\to a_t=l\omega\]
\[a_t=l\frac{d^2\theta}{dt^2}\to l\frac{d^2\theta}{dt^2}=-g\sin\theta\]\\
\[\boxed{\therefore\quad\frac{d^2\theta}{dt^2}+\frac{g}{l}\sin\theta=0}\]

\newpage
\LARGE
Formulación de Euler-Lagrange\\\\
\Large
\[\frac{d}{dt}\left[\frac{\partial\mathcal{L}}{\partial\dot q}\right]-\frac{\partial\mathcal{L}}{\partial q}=0\quad,\quad\mathcal{L}=T-V\]\\
Coordenadas generalizadas
\[q_1=\theta\]\\
\[x=l\sin\theta\quad,\quad\dot x=l\dot\theta\cos\theta\]
\[y=-l\cos\theta\quad,\quad\dot y=l\dot\theta\sin\theta\]\\
Energía potencial
\[V=mgy\to V=-mgl\cos\theta\]\\
Energía cinética
\[T=\frac{1}{2}m(\dot x^2+\dot y^2)\]
\[T=\frac{1}{2}m[(l\dot\theta\cos\theta)^2+(l\dot\theta\sin\theta)^2]\to T=\frac{1}{2}m[l^2\dot\theta^2\cos^2\theta+l^2\dot\theta^2\sin^2\theta]\]
\[T=\frac{1}{2}ml^2\dot\theta^2\]\\
Lagrangiano
\[\mathcal{L}=T-V\]
\[\mathcal{L}=\frac{1}{2}ml^2\dot\theta^2+mgl\cos\theta\]\\
Derivadas del Lagrangiano
\[\frac{\partial\mathcal{L}}{\partial\theta}=-mgl\sin\theta\]
\[\frac{\partial\mathcal{L}}{\partial\dot\theta}=ml^2\dot\theta\]
\[\frac{d}{dt}\left[\frac{\partial\mathcal{L}}{\partial\dot\theta}\right]=ml^2\ddot\theta\]

\newpage
Ecuación de movimiento
\[\frac{d}{dt}\left[\frac{\partial\mathcal{L}}{\partial\dot\theta}\right]-\frac{\partial\mathcal{L}}{\partial\theta}=0\]
\[ml^2\ddot\theta+mgl\sin\theta=0\to l^2\ddot\theta+gl\sin\theta=0\]\\
\[\boxed{\therefore\quad\ddot\theta+\frac{g}{l}\sin\theta=0}\]

\newpage
Solución analítica aproximada para ángulos pequeños
\[\theta\ll\frac{\pi}{2}\to\sin\theta\approx\theta\]
\[\therefore\quad\frac{g}{l}\sin\theta\approx\frac{g}{l}\theta\]\\
\[\ddot\theta+\frac{g}{l}\sin\theta=0\to\quad\ddot\theta+\frac{g}{l}\theta=0\]\\
Ecuación no lineal de segundo orden
\[\ddot\theta+\frac{g}{l}\theta=0\]
\[\omega=\sqrt\frac{g}{l}\]\\
\[\theta=e^{\beta t}\]
\[\dot\theta=\beta e^{\beta t}\]
\[\ddot\theta=\beta^2e^{\beta t}\]\\
\[\ddot\theta+\frac{g}{l}\theta=0\to\beta^2e^{\beta t}+\omega^2e^{\beta t}=0\]
\[\beta^2=-\omega^2\to\beta=\sqrt{-\omega^2}\to\beta=\omega\sqrt{-1}\]
\[\beta=\pm i\omega\]

\newpage
\[\theta=C_1e^{i\omega t}+C_2e^{-i\omega t}\]
\[\theta=C_1e^{i\omega t}+C_2e^{-i\omega t}\]
\[e^{\pm ik}=\cos k\pm i\sin k\]
\[\theta=C_1\cos(\omega t)+iC_1\sin(\omega t)+C_2\cos(\omega t)-iC_2\sin(\omega t)\]
\[\theta=(C_1+C_2)\cos(\omega t)+(C_1-C_2)i\sin(\omega t)\]
\[A=C_1+C_2\quad,\quad B=i(C_1-C_2)\]\\
\[\theta=A\cos(\omega t)+B\sin(\omega t)\]
\[\dot\theta=-A\omega\sin(\omega t)+B\omega\cos(\omega t)\]\\
\[\theta(t=0)=\theta_0\quad,\quad\theta(t=0)=A\cancelto{1}{\cos(\omega(0))}+B\cancelto{0}{\sin(\omega(0))}\to A=\theta_0\]
\[\dot\theta(t=0)=0\quad,\quad\dot\theta(t=0)=-A\cancelto{0}{\omega\sin(\omega(0))}+B\cancelto{1}{\omega\cos(\omega(0))}\to B=0\]\\
\[\therefore\quad\theta=Acos(\phi+\omega t)\]\\
\[\boxed{\theta(t)=Acos\left(\phi+\sqrt\frac{g}{l}t\right)}\]

\newpage
Solución númerica por método de Euler
\[x_{i+1}=x_i+h\dot x(x_i,t_i)\]\\
\[\ddot\theta+\frac{g}{l}\sin\theta=0\]
\[\omega=\dot\theta\to\dot\omega=\ddot\theta\]
\[\dot\omega=-\frac{g}{l}\sin\theta\]\\
\[\boxed{
  \begin{array}{rcl}
    t_{i+1}=t_i+h\\
    \theta_{i+1}=\theta_i+h\omega_{i+1}\\
    \omega_{i+1}=\omega_i-h\frac{g}{l}\sin\theta_i
  \end{array}
}
\]\\

\end{document}
