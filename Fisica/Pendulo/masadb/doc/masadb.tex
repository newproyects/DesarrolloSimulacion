\documentclass[10pt,a4papper]{article}
\usepackage{graphicx}
\usepackage{amsmath}
\usepackage{amssymb}
\usepackage{cancel}
\usepackage[hidelinks]{hyperref}
\usepackage[left=1.50cm, right=3.00cm, top=2.00cm, bottom=2.00cm]{geometry}
\author{Angel Fdo. García Núñez}
\date{Septiembre 17, 2020}
\title{Estadisitica}

\begin{document}

\Huge
Desarrollo y simulación de péndulo de masa doble\\

Angel Fernando García Núñez

\newpage
\LARGE
Formulación de Euler-Lagrange\\\\
\Large
\[\frac{d}{dt}\left[\frac{\partial\mathcal{L}}{\partial\dot q}\right]-\frac{\partial\mathcal{L}}{\partial q}=0\quad,\quad\mathcal{L}=T-V\]\\
Coordenadas generalizadas
\[q_1=\theta_1\]
\[q_2=\theta_2\]\\
\[S_1=2l_1\quad,\quad S_2=l_2\]
\[x_1=\frac{S_1}{2}\sin\theta_1\quad,\quad\dot x_1=\frac{S_1}{2}\dot\theta_1\cos\theta_1\]
\[y_1=-\frac{S_1}{2}\cos\theta_1\quad,\quad\dot y_1=\frac{S_1}{2}\dot\theta_1\sin\theta_1\]\\
\[x_2=2x_1+l_2\sin\theta_2\to x_2=S_1\sin\theta_1+l_2\sin\theta_2\quad,\quad\dot x_2=S_1\dot\theta_1\cos\theta_1+l_2\dot\theta_2\cos\theta_2\]
\[y_2=2y_1-l_2\cos\theta_2\to y_2=-S_1\cos\theta_1-l_2\cos\theta_2\quad,\quad\dot y_2=S_1\dot\theta_1\sin\theta_1+l_2\dot\theta_2\sin\theta_2\]\\
Energía potencial
\[V=m_1gy_1+m_2gy_2\to V=-m_1g\frac{S_1}{2}\cos\theta_1+m_2g(-S_1\cos\theta_1-l_2\cos\theta_2)\]\\
\[V=-m_1g\frac{S_1}{2}\cos\theta_1-m_2gS_1\cos\theta_1-m_2gl_2\cos\theta_2\]\\
\[\mu=\frac{1}{2}m_1+m_2\]\\
\[V=-\mu gS_1\cos\theta_1-m_2gl_2\cos\theta_2\]

\newpage
Energía cinética
\[T=\frac{1}{2}I_1\dot\theta_1^2+\frac{1}{2}I_2\dot\theta_2^2+\frac{1}{2}m_2(\dot x_2+\dot y_2)\]\\
\[\dot x_2^2=(S_1\dot\theta_1\cos\theta_1+l_2\dot\theta_2\cos\theta_2)^2\to\dot x_2^2=S_1^2\dot\theta_1^2\cos^2\theta_1+l_2^2\dot\theta_2^2\cos^2\theta_2+2S_1l_2\dot\theta_1\dot\theta_2\cos\theta_1\cos\theta_2\]
\[\dot y_2^2=(S_1\dot\theta_1\sin\theta_1+l_2\dot\theta_2\sin\theta_2)^2\to\dot y_2^2=S_1^2\dot\theta_1^2\sin^2\theta_1+l_2^2\dot\theta_2^2\sin^2\theta_2+2S_1l_2\dot\theta_1\dot\theta_2\sin\theta_1\sin\theta_2\]\\

\[T=\frac{1}{2}I_1\dot\theta_1^2+\frac{1}{2}I_2\dot\theta_2^2+\]
\[\frac{1}{2}m_2\left(S_1^2\dot\theta_1^2\cos^2\theta_1+l_2^2\dot\theta_2^2\cos^2\theta_2+2S_1l_2\dot\theta_1\dot\theta_2\cos\theta_1\cos\theta_2+\right\]
\[\left S_1^2\dot\theta_1^2\sin^2\theta_1+l_2^2\dot\theta_2^2\sin^2\theta_2+2S_1l_2\dot\theta_1\dot\theta_2\sin\theta_1\sin\theta_2\right)\]\\

\[T=\frac{1}{2}I_1\dot\theta_1^2+\frac{1}{2}I_2\dot\theta_2^2+\]
\[\frac{1}{2}m_2\left(S_1^2\dot\theta_1^2+l_2^2\dot\theta_2^2+2S_1l_2\dot\theta_1\dot\theta_2\cos\theta_1\cos\theta_2+2S_1l_2\dot\theta_1\dot\theta_2\sin\theta_1\sin\theta_2\right)\]


\newpage
Energía cinética\\
Producto de cosenos
\[\cos a\cos b=\frac{1}{2}\left[cos(a+b)+cos(a-b)\right]\]
Producto de senos
\[\sin a\sin b=\frac{1}{2}\left[cos(a-b)-cos(a+b)\right]\]\\
\[T=\frac{1}{2}I_1\dot\theta_1^2+\frac{1}{2}I_2\dot\theta_2^2+\frac{1}{2}m_2\left(S_1^2\dot\theta_1^2+l_2^2\dot\theta_2^2+2S_1l_2\dot\theta_1\dot\theta_2\cos(\theta_1-\theta_2)\right)\]\\

\[T=\frac{1}{2}I_1\dot\theta_1^2+\frac{1}{2}I_2\dot\theta_2^2+\frac{1}{2}m_2S_1^2\dot\theta_1^2+\frac{1}{2}m_2l_2^2\dot\theta_2^2+m_2S_1l_2\dot\theta_1\dot\theta_2\cos(\theta_1-\theta_2)\]

\newpage
Lagrangiano
\[\mathcal{L}=T-V\]\\
\[\mathcal{L}=\frac{1}{2}I_1\dot\theta_1^2+\frac{1}{2}I_2\dot\theta_2^2+\frac{1}{2}m_2S_1^2\dot\theta_1^2+\frac{1}{2}m_2l_2^2\dot\theta_2^2+m_2S_1l_2\dot\theta_1\dot\theta_2\cos(\theta_1-\theta_2)+\mu gS_1\cos\theta_1+m_2gl_2\cos\theta_2\]\\
Derivadas del Lagrangiano
\[\frac{\partial\mathcal{L}}{\partial\theta_1}=-m_2S_1l_2\dot\theta_1\dot\theta_2\sin(\theta_1-\theta_2)-\mu gS_1\sin\theta_1\]
\[\frac{\partial\mathcal{L}}{\partial\dot\theta_1}=I_1\dot\theta_1+m_2S_1^2\dot\theta_1+m_2S_1l_2\dot\theta_2\cos(\theta_1-\theta_2)\]
\[\frac{d}{dt}\left[\frac{\partial\mathcal{L}}{\partial\dot\theta_1}\right]=I_1\ddot\theta_1+m_2S_1^2\ddot\theta_1+m_2S_1l_2\ddot\theta_2\cos(\theta_1-\theta_2)+m_2S_1l_2\dot\theta_2^2\sin(\theta_1-\theta_2)-m_2S_1l_2\dot\theta_1\dot\theta_2\sin(\theta_1-\theta_2)\]\\

\[\frac{\partial\mathcal{L}}{\partial\theta_2}=m_2S_1l_2\dot\theta_1\dot\theta_2\sin(\theta_1-\theta_2)-m_2gl_2\sin\theta_2\]
\[\frac{\partial\mathcal{L}}{\partial\dot\theta_2}=I_2\dot\theta_2+m_2l_2^2\dot\theta_2+m_2S_1l_2\dot\theta_1\cos(\theta_1-\theta_2)\]
\[\frac{d}{dt}\left[\frac{\partial\mathcal{L}}{\partial\dot\theta_2}\right]=I_2\ddot\theta_2+m_2l_2^2\ddot\theta_2+m_2S_1l_2\ddot\theta_1\cos(\theta_1-\theta_2)-m_2S_1l_2\dot\theta_1^2\sin(\theta_1-\theta_2)+m_2S_1l_2\dot\theta_1\dot\theta_2\sin(\theta_1-\theta_2)\]

\newpage
Ecuaciones de Lagrange
\[\frac{d}{dt}\left[\frac{\partial\mathcal{L}}{\partial\dot \theta_1}\right]-\frac{\partial\mathcal{L}}{\partial \theta_1}=0\]\\

\[I_1\ddot\theta_1+m_2S_1^2\ddot\theta_1+m_2S_1l_2\ddot\theta_2\cos(\theta_1-\theta_2)+m_2S_1l_2\dot\theta_2^2\sin(\theta_1-\theta_2)-m_2S_1l_2\dot\theta_1\dot\theta_2\sin(\theta_1-\theta_2)+\]
\[m_2S_1l_2\dot\theta_1\dot\theta_2\sin(\theta_1-\theta_2)+\mu gS_1\sin\theta_1=0\]\\
\[I_1\ddot\theta_1+m_2S_1^2\ddot\theta_1+m_2S_1l_2\ddot\theta_2\cos(\theta_1-\theta_2)+m_2S_1l_2\dot\theta_2^2\sin(\theta_1-\theta_2)+\mu gS_1\sin\theta_1=0\]\\

\[\frac{d}{dt}\left[\frac{\partial\mathcal{L}}{\partial\dot \theta_2}\right]-\frac{\partial\mathcal{L}}{\partial \theta_2}=0\]\\

\[I_2\ddot\theta_2+m_2l_2^2\ddot\theta_2+m_2S_1l_2\ddot\theta_1\cos(\theta_1-\theta_2)-m_2S_1l_2\dot\theta_1^2\sin(\theta_1-\theta_2)+m_2S_1l_2\dot\theta_1\dot\theta_2\sin(\theta_1-\theta_2)-\]
\[m_2S_1l_2\dot\theta_1\dot\theta_2\sin(\theta_1-\theta_2)+m_2gl_2\sin\theta_2=0\]\\
\[I_2\ddot\theta_2+m_2l_2^2\ddot\theta_2+m_2S_1l_2\ddot\theta_1\cos(\theta_1-\theta_2)-m_2S_1l_2\dot\theta_1^2\sin(\theta_1-\theta_2)+m_2gl_2\sin\theta_2=0\]

\newpage
Ecuaciones de movimiento\\\\
\[\boxed{
  \begin{array}{rcl}
    \ddot\theta_1+(I_1+m_2S_1^2)^{-1}\left[m_2S_1l_2\ddot\theta_2\cos(\theta_1-\theta_2)+m_2S_1l_2\dot\theta_2^2\sin(\theta_1-\theta_2)+\mu gS_1\sin\theta_1\right]=0\\\\
    \ddot\theta_2+(I_2+m_2l_2^2)^{-1}\left[m_2S_1l_2\ddot\theta_1\cos(\theta_1-\theta_2)-m_2S_1l_2\dot\theta_1^2\sin(\theta_1-\theta_2)+m_2gl_2\sin\theta_2\right]=0
  \end{array}
}\]

\newpage
Solución numerica por\\
Metodo de Runge Kutta de orden 4\\\\
\large
\begin{center}
  \begin{tabular}{|c|c|}
    \hline
    $\frac{d\theta_1}{dt}=\omega_1$ & $\frac{d\omega_1}{dt}=f(\theta_1,\theta_2,\omega_1,\omega_2,t)$\\
    \hline
    $k_1=h\omega_1$ & $c_1=hf(\theta_1,\theta_2,\omega_1,\omega_2,t)$\\
    $k_2=h(\omega_1+\frac{1}{2}c_1)$ & $c_2=hf(\theta_1+\frac{1}{2}k_1,\theta_2+\frac{1}{2}n_1,\omega_1+\frac{1}{2}c_1,\omega_2+\frac{1}{2}m_1,t+\frac{1}{2}h)$\\
    $k_3=h(\omega_1+\frac{1}{2}c_2)$ & $c_3=hf(\theta_1+\frac{1}{2}k_2,\theta_2+\frac{1}{2}n_2,\omega_1+\frac{1}{2}c_2,\omega_2+\frac{1}{2}m_2,t+\frac{1}{2}h)$\\
    $k_4=h(\omega_1+c_3)$ & $c_4=hf(\theta_1+k_3,\theta_2+n_3,\omega_1+c_3,\omega_2+m_3,t+h)$\\
    \hline
    $\theta_1(t+h)=\theta_1(t)+\frac{1}{6}(k_1+2k_2+2k_3+k_4)$ & $\omega_1(t+h)=\omega_1(t)+\frac{1}{6}(c_1+2c_2+2c_3+c_4)$\\
    \hline
  \end{tabular}
\end{center}
\begin{center}
  \begin{tabular}{|c|c|}
    \hline
    $\frac{d\theta_2}{dt}=\omega_2$ & $\frac{d\omega_2}{dt}=u(\theta_1,\theta_2,\omega_1,\omega_2,t)$\\
    \hline
    $n_1=h\omega_2$ & $m_1=hu(\theta_1,\theta_2,\omega_1,\omega_2,t)$\\
    $n_2=h(\omega_2+\frac{1}{2}m_1)$ & $m_2=hu(\theta_1+\frac{1}{2}k_1,\theta_2+\frac{1}{2}n_1,\omega_1+\frac{1}{2}c_1,\omega_2+\frac{1}{2}m_1,t+\frac{1}{2}h)$\\
    $n_3=h(\omega_2+\frac{1}{2}m_2)$ & $m_3=hu(\theta_1+\frac{1}{2}k_2,\theta_2+\frac{1}{2}n_2,\omega_1+\frac{1}{2}c_2,\omega_2+\frac{1}{2}m_2,t+\frac{1}{2}h)$\\
    $n_4=h(\omega_2+m_3)$ & $m_4=hu(\theta_1+k_3,\theta_2+n_3,\omega_1+c_3,\omega_2+m_3,t+h)$\\
    \hline
    $\theta_2(t+h)=\theta_2(t)+\frac{1}{6}(n_1+2n_2+2n_3+n_4)$ & $\omega_2(t+h)=\omega_2(t)+\frac{1}{6}(m_1+2m_2+2m_3+m_4)$\\
    \hline
  \end{tabular}
\end{center}

\newpage
\Large
Cambio de variable\\
\[\frac{d\theta_1}{dt}=\omega_1\to\frac{d^2\theta_1}{dt^2}=\dot\omega_1\quad,\quad\frac{d\theta_2}{dt}=\omega_2\to\frac{d^2\theta_2}{dt^2}=\dot\omega_2\]\\

\[\dot\omega_1=-(I_1+m_2S_1^2)^{-1}\left[m_2S_1l_2\dot\omega_2\cos(\theta_1-\theta_2)+m_2S_1l_2\omega_2^2\sin(\theta_1-\theta_2)+\mu gS_1\sin\theta_1\right]\]
\[\dot\omega_2=-(I_2+m_2l_2^2)^{-1}\left[m_2S_1l_2\dot\omega_1\cos(\theta_1-\theta_2)-m_2S_1l_2\omega_1^2\sin(\theta_1-\theta_2)+m_2gl_2\sin\theta_2\right]\]\\

%\[\dot\omega_1=\left[1-m_2^2S_1^2l_2^2\cos^2(\theta_1-\theta_2)(I_1+m_2S_1^2)^{-1}(I_2+m_2l_2^2)^{-1}\right]^{-1}\]
%\[\left\{m_2S_1l_2\cos(\theta_1-\theta_2)(I_1+m_2S_1^2)^{-1}(I_2+m_2l_2^2)^{-1}\left[-m_2S_1l_2\omega_1^2\sin(\theta_1-\theta_2)+m_2gl_2\sin\theta_2\right]\]
%\[\left+(I_1+m_2S_1^2)^{-1}\left[m_2S_1l_2\omega_2^2\sin(\theta_1-\theta_2)+\mu gS_1\sin\theta_1\right]\right\}\]

Simplificación y despeje
\[ms=m_2S_2^2\]
\[ml=m_2l_2^2\]
\[msl=m_2S_2l_2\]
\[i_1=(I_1+m_2S_1^2)^{-1}\]
\[i_2=(I_2+m_2l_2^2)^{-1}\]
\[\varepsilon_1=m_2S_1l_2\omega_2^2\sin(\theta_1-\theta_2)+\mu gS_1\sin\theta_1\]
\[\varepsilon_2=-msl\omega_1^2\sin(\theta_1-\theta_2)+m_2gl_2\sin\theta_2\]\\

\[\dot\omega_1=-i_1\left[msl\dot\omega_2\cos(\theta_1-\theta_2)+\varepsilon_1\right]\]
\[\dot\omega_2=-i_2\left[msl\dot\omega_1\cos(\theta_1-\theta_2)+\varepsilon_2\right]\]\\

\[\dot\omega_1=-i_1msl\dot\omega_2\cos(\theta_1-\theta_2)-i_1\varepsilon_1\]
\[\dot\omega_1=i_1i_2msl\cos(\theta_1-\theta_2)\left[msl\dot\omega_1\cos(\theta_1-\theta_2)+\varepsilon_2\right]-i_1\varepsilon_1\]
\[\dot\omega_1=i_1i_2(msl)^2\dot\omega_1\cos^2(\theta_1-\theta_2)+i_1i_2msl\cos(\theta_1-\theta_2)\varepsilon_2-i_1\varepsilon_1\]
\[\dot\omega_1[1-i_1i_2(msl)^2\cos^2(\theta_1-\theta_2)]=i_1i_2msl\cos(\theta_1-\theta_2)\varepsilon_2-i_1\varepsilon_1\]\\

\[\dot\omega_1=\frac{[i_1i_2msl\cos(\theta_1-\theta_2)\varepsilon_2-i_1\varepsilon_1]}{[1-i_1i_2(msl)^2\cos^2(\theta_1-\theta_2)]}\]


\newpage
\[\dot\omega_1=\frac{(I_1+m_2S_1^2)^{-1}(I_2+m_2l_2^2)^{-1}m_2S_2l_2\cos(\theta_1-\theta_2)[-msl\omega_1^2\sin(\theta_1-\theta_2)+m_2gl_2\sin\theta_2]}{[1-(I_1+m_2S_1^2)^{-1}(I_2+m_2l_2^2)^{-1}m_2^2S_2^2l_2^2\cos^2(\theta_1-\theta_2)]}\]
\[-\frac{(I_1+m_2S_1^2)^{-1}[m_2S_1l_2\omega_2^2\sin(\theta_1-\theta_2)+\mu gS_1\sin\theta_1]}{[1-(I_1+m_2S_1^2)^{-1}(I_2+m_2l_2^2)^{-1}m_2^2S_2^2l_2^2\cos^2(\theta_1-\theta_2)]}\]\\

Funciones de iteración
\[\boxed{
  \begin{array}{lcl}
    f(\theta_1,\theta_2,\omega_1,\omega_2,t)=\frac{d\omega_1}{dt}\\
    f(\theta_1,\theta_2,\omega_1,\omega_2,t)=\\\\
    \frac{(I_1+m_2S_1^2)^{-1}(I_2+m_2l_2^2)^{-1}m_2S_2l_2\cos(\theta_1-\theta_2)[-msl\omega_1^2\sin(\theta_1-\theta_2)+m_2gl_2\sin\theta_2]}{[1-(I_1+m_2S_1^2)^{-1}(I_2+m_2l_2^2)^{-1}m_2^2S_2^2l_2^2\cos^2(\theta_1-\theta_2)]}\\\\
    -\frac{(I_1+m_2S_1^2)^{-1}[m_2S_1l_2\omega_2^2\sin(\theta_1-\theta_2)+\mu gS_1\sin\theta_1]}{[1-(I_1+m_2S_1^2)^{-1}(I_2+m_2l_2^2)^{-1}m_2^2S_2^2l_2^2\cos^2(\theta_1-\theta_2)]}\\\\\\
    
    u(\theta_1,\theta_2,\omega_1,\omega_2,t)=\frac{d\omega_2}{dt}\\
    u(\theta_1,\theta_2,\omega_1,\omega_2,t)=-(I_2+m_2l_2^2)^{-1}\left[m_2S_1l_2\cos(\theta_1-\theta_2)f(\theta_1,\theta_2,\omega_1,\omega_2,t)\\
      \left-m_2S_1l_2\omega_1^2\sin(\theta_1-\theta_2)+m_2gl_2\sin\theta_2\right]
  \end{array}
}\]



\end{document}
