\documentclass[10pt,a4papper]{article}
\usepackage{graphicx}
\usepackage{amsmath}
\usepackage{amssymb}
\usepackage{cancel}
\usepackage[hidelinks]{hyperref}
\usepackage[left=1.00cm, right=3.00cm, top=2.00cm, bottom=2.00cm]{geometry}
\author{Angel Fernando García Núñez}
\date{Septiembre 17, 2020}
\title{Estadisitica}

\begin{document}

\Huge
Desarrollo y simulación de péndulo simple doble\\

Angel Fernando García Núñez

\newpage
\LARGE
Formulación de Euler-Lagrange\\\\
\Large
\[\frac{d}{dt}\left[\frac{\partial\mathcal{L}}{\partial\dot q}\right]-\frac{\partial\mathcal{L}}{\partial q}=0\quad,\quad\mathcal{L}=T-V\]\\
Coordenadas generalizadas
\[q_1=\theta_1\]
\[q_2=\theta_2\]\\
\[x_1=l_1\sin\theta_1\quad,\quad\dot x_1=l_1\dot\theta_1\cos\theta_1\]
\[y_1=-l_1\cos\theta_1\quad,\quad\dot y_1=l_1\dot\theta_1\sin\theta_1\]\\
\[x_2=x_1+l_2\sin\theta_2\to x_2=l_1\sin\theta_1+l_2\sin\theta_2\quad,\quad\dot x_2=l_1\dot\theta_1\cos\theta_1+l_2\dot\theta_2\cos\theta_2\]
\[y_2=y_1-l_2\cos\theta_2\to y_2=-l_1\cos\theta_1-l_2\cos\theta_2\quad,\quad\dot y_2=l_1\dot\theta_1\sin\theta_1+l_2\dot\theta_2\sin\theta_2\]\\
Energía potencial
\[V=m_1gy_1+m_2gy_2\to V=-m_1gl_1\cos\theta_1+m_2g(-l_1\cos\theta_1-l_2\cos\theta_2)\]\\
\[\mu=m_1+m_2\quad,\quad L=\frac{l_2}{l_1}\]\\
\[V=-g\mu l_1\cos\theta_1-gm_2l_2\cos\theta_2\]

\newpage
Energía cinética
\[T=\frac{1}{2}m_1(\dot x_1^2+\dot y_1^2)+\frac{1}{2}m_2(\dot x_2^2+\dot y_2^2)\]
\[\dot x_1^2=l_1^2\dot\theta_1^2\cos^2\theta_1\]
\[y_1^2=l_1^2\dot\theta_1^2\sin^2\theta_1\]
\[\dot x_2^2=(l_1\dot\theta_1\cos\theta_1+l_2\dot\theta_2\cos\theta_2)^2\to\dot x_2^2=l_1^2\dot\theta_1^2\cos^2\theta_1+l_2^2\dot\theta_2^2\cos^2\theta_2+2l_1l_2\dot\theta_1\dot\theta_2\cos\theta_1\cos\theta_2\]
\[\dot y_2^2=(l_1\dot\theta_1\sin\theta_1+l_2\dot\theta_2\sin\theta_2)^2\to\dot y_2^2=l_1^2\dot\theta_1^2\sin^2\theta_1+l_2^2\dot\theta_2^2\sin^2\theta_2+2l_1l_2\dot\theta_1\dot\theta_2\sin\theta_1\sin\theta_2\]\\
\[T=\frac{1}{2}m_1l_1^2\dot\theta_1^2+\frac{1}{2}m_2l_1^2\dot\theta_1^2+\frac{1}{2}m_2l_2^2\dot\theta_2^2+m_2l_1l_2\dot\theta_1\dot\theta_2\cos\theta_1\cos\theta_2+m_2l_1l_2\dot\theta_1\dot\theta_2\sin\theta_1\sin\theta_2\]\\
Producto de cosenos
\[\cos a\cos b=\frac{1}{2}\left[cos(a+b)+cos(a-b)\right]\]
Producto de senos
\[\sin a\sin b=\frac{1}{2}\left[cos(a-b)-cos(a+b)\right]\]\\
\[T=\frac{1}{2}m_1l_1^2\dot\theta_1^2+\frac{1}{2}m_2l_1^2\dot\theta_1^2+\frac{1}{2}m_2l_2^2\dot\theta_2^2+m_2l_1l_2\dot\theta_1\dot\theta_2\cos(\theta_1-\theta_2)\]

\newpage
Lagrangiano
\[\mathcal{L}=T-V\]
\[\mathcal{L}=\frac{1}{2}m_1l_1^2\dot\theta_1^2+\frac{1}{2}m_2l_1^2\dot\theta_1^2+\frac{1}{2}m_2l_2^2\dot\theta_2^2+m_2l_1l_2\dot\theta_1\dot\theta_2\cos(\theta_1-\theta_2)+g\mu l_1\cos\theta_1+gm_2l_2\cos\theta_2\]\\\\
Derivadas del Lagrangiano
\[\frac{\partial\mathcal{L}}{\partial\theta_1}=-m_2l_1l_2\dot\theta_1\dot\theta_2\sin(\theta_1-\theta_2)-g\mu l_1\sin\theta_1\]
\[\frac{\partial\mathcal{L}}{\partial\dot\theta_1}=\mu l_1^2\dot\theta_1+m_2l_1l_2\dot\theta_2\cos(\theta_1-\theta_2)\]
\[\frac{d}{dt}\left[\frac{\partial\mathcal{L}}{\partial\dot\theta_1}\right]=\mu l_1^2\ddot\theta_1+m_2l_1l_2\ddot\theta_2\cos(\theta_1-\theta_2)-m_2l_1l_2\dot\theta_1\dot\theta_2\sin(\theta_1-\theta_2)+m_2l_1l_2\dot\theta_2^2\sin(\theta_1-\theta_2)\]\\
\[\frac{\partial\mathcal{L}}{\partial\theta_2}=m_2l_1l_2\dot\theta_1\dot\theta_2\sin(\theta_1-\theta_2)-gm_2l_2\sin\theta_2\]
\[\frac{\partial\mathcal{L}}{\partial\dot\theta_2}=m_2l_2^2\dot\theta_2+m_2l_1l_2\dot\theta_1\cos(\theta_1-\theta_2)\]
\[\frac{d}{dt}\left[\frac{\partial\mathcal{L}}{\partial\dot\theta_2}\right]=m_2l_2^2\ddot\theta_2+m_2l_1l_2\ddot\theta_1\cos(\theta_1-\theta_2)+m_2l_1l_2\dot\theta_1\dot\theta_2\sin(\theta_1-\theta_2)-m_2l_1l_2\dot\theta_1^2\sin(\theta_1-\theta_2)\]\\

\newpage
Ecuaciones de Lagrange
\[\frac{d}{dt}\left[\frac{\partial\mathcal{L}}{\partial\dot \theta_1}\right]-\frac{\partial\mathcal{L}}{\partial \theta_1}=0\]\\
\[\mu l_1^2\ddot\theta_1+m_2l_1l_2\ddot\theta_2\cos(\theta_1-\theta_2)+m_2l_1l_2\dot\theta_2^2\sin(\theta_1-\theta_2)+g\mu l_1\sin\theta_1=0\]
\[\]\\\\
\[\frac{d}{dt}\left[\frac{\partial\mathcal{L}}{\partial\dot \theta_2}\right]-\frac{\partial\mathcal{L}}{\partial \theta_2}=0\]\\
\[l_2^2\ddot\theta_2+l_1l_2\ddot\theta_1\cos(\theta_1-\theta_2)-l_1l_2\dot\theta_1^2\sin(\theta_1-\theta_2)+gl_2\sin\theta_2=0\]\\\\\\\\\\\\\\\\\\\\\\\\\\\\\\\\
\[\mu=m_1+m2\quad,\quad L=\frac{l_2}{l_1}\]


\newpage
Ecuaciones de movimiento\\\\
\[\boxed{
  \begin{array}{rcl}
    \ddot\theta_1+\frac{m_2}{\mu}L\cos(\theta_1-\theta_2)\ddot\theta_2+\frac{m_2}{\mu}L\sin(\theta_1-\theta_2)\dot\theta_2^2+\frac{g}{l_1}\sin\theta_1=0\\\\
    \ddot\theta_2+L^{-1}\cos(\theta_1-\theta_2)\ddot\theta_1-L^{-1}\sin(\theta_1-\theta_2)\dot\theta_1^2+\frac{g}{l_2}\sin\theta_2=0
  \end{array}
}
\]

\newpage
Solución numerica por\\
Metodo de Runge Kutta de orden 4\\\\
\large
\begin{center}
  \begin{tabular}{|c|c|}
    \hline
    $\frac{d\theta_1}{dt}=\omega_1$ & $\frac{d\omega_1}{dt}=f(\theta_1,\theta_2,\omega_1,\omega_2,t)$\\
    \hline
    $k_1=h\omega_1$ & $c_1=hf(\theta_1,\theta_2,\omega_1,\omega_2,t)$\\
    $k_2=h(\omega_1+\frac{1}{2}c_1)$ & $c_2=hf(\theta_1+\frac{1}{2}k_1,\theta_2+\frac{1}{2}n_1,\omega_1+\frac{1}{2}c_1,\omega_2+\frac{1}{2}m_1,t+\frac{1}{2}h)$\\
    $k_3=h(\omega_1+\frac{1}{2}c_2)$ & $c_3=hf(\theta_1+\frac{1}{2}k_2,\theta_2+\frac{1}{2}n_2,\omega_1+\frac{1}{2}c_2,\omega_2+\frac{1}{2}m_2,t+\frac{1}{2}h)$\\
    $k_4=h(\omega_1+c_3)$ & $c_4=hf(\theta_1+k_3,\theta_2+n_3,\omega_1+c_3,\omega_2+m_3,t+h)$\\
    \hline
    $\theta_1(t+h)=\theta_1(t)+\frac{1}{6}(k_1+2k_2+2k_3+k_4)$ & $\omega_1(t+h)=\omega_1(t)+\frac{1}{6}(c_1+2c_2+2c_3+c_4)$\\
    \hline
  \end{tabular}
\end{center}
\begin{center}
  \begin{tabular}{|c|c|}
    \hline
    $\frac{d\theta_2}{dt}=\omega_2$ & $\frac{d\omega_2}{dt}=u(\theta_1,\theta_2,\omega_1,\omega_2,t)$\\
    \hline
    $n_1=h\omega_2$ & $m_1=hu(\theta_1,\theta_2,\omega_1,\omega_2,t)$\\
    $n_2=h(\omega_2+\frac{1}{2}m_1)$ & $m_2=hu(\theta_1+\frac{1}{2}k_1,\theta_2+\frac{1}{2}n_1,\omega_1+\frac{1}{2}c_1,\omega_2+\frac{1}{2}m_1,t+\frac{1}{2}h)$\\
    $n_3=h(\omega_2+\frac{1}{2}m_2)$ & $m_3=hu(\theta_1+\frac{1}{2}k_2,\theta_2+\frac{1}{2}n_2,\omega_1+\frac{1}{2}c_2,\omega_2+\frac{1}{2}m_2,t+\frac{1}{2}h)$\\
    $n_4=h(\omega_2+m_3)$ & $m_4=hu(\theta_1+k_3,\theta_2+n_3,\omega_1+c_3,\omega_2+m_3,t+h)$\\
    \hline
    $\theta_2(t+h)=\theta_2(t)+\frac{1}{6}(n_1+2n_2+2n_3+n_4)$ & $\omega_2(t+h)=\omega_2(t)+\frac{1}{6}(m_1+2m_2+2m_3+m_4)$\\
    \hline
  \end{tabular}
\end{center}
\Large



\newpage
Cambio de variable
\[\frac{d\theta_1}{dt}=\omega_1\to\frac{d^2\theta_1}{dt^2}=\dot\omega_1\quad,\quad\frac{d\theta_2}{dt}=\omega_2\to\frac{d^2\theta_2}{dt^2}=\dot\omega_2\]\\
\[\dot\omega_1=-\frac{m_2}{\mu}L\cos(\theta_1-\theta_2)\dot\omega_2-\frac{m_2}{\mu}L\sin(\theta_1-\theta_2)\omega_2^2-\frac{g}{l_1}\sin\theta_1\]\\
\[\dot\omega_2=-L^{-1}\cos(\theta_1-\theta_2)\dot\omega_1+L^{-1}\sin(\theta_1-\theta_2)\omega_1^2-\frac{g}{l_2}\sin\theta_2\]\\
\[\dot\omega_1=\frac{\frac{g}{l_1}\left[\frac{m_2}{\mu}\cos(\theta_1-\theta_2)\sin\theta_2-\sin\theta_1\right]-\frac{m_2}{\mu}\sin(\theta_1-\theta_2)\left[\cos(\theta_1-\theta_2)\omega_1^2+L\omega_2^2\right]}{\left[1-\frac{m_2}{\mu}\cos^2(\theta_1-\theta_2)\right]}\]\\

\newpage
\Large
Funciones de iteración
\[\boxed{
  \begin{array}{lcl}
    f(\theta_1,\theta_2,\omega_1,\omega_2,t)=\frac{d\omega_1}{dt}\\
    f(\theta_1,\theta_2,\omega_1,\omega_2,t)=\frac{\frac{g}{l_1}\left[\frac{m_2}{\mu}\cos(\theta_1-\theta_2)\sin\theta_2-\sin\theta_1\right]-\frac{m_2}{\mu}\sin(\theta_1-\theta_2)\left[\cos(\theta_1-\theta_2)\omega_1^2+L\omega_2^2\right]}{\left[1-\frac{m_2}{\mu}\cos^2(\theta_1-\theta_2)\right]}\\\\
    u(\theta_1,\theta_2,\omega_1,\omega_2,t)=\frac{d\omega_2}{dt}\\
    u(\theta_1,\theta_2,\omega_1,\omega_2,t)=-L^{-1}\cos(\theta_1-\theta_2)f(\theta_1,\theta_2,\omega_1,\omega_2,t)+L^{-1}\sin(\theta_1-\theta_2)\omega_1^2-\frac{g}{l_2}\sin\theta_2
  \end{array}
}
\]\\\\


\end{document}
