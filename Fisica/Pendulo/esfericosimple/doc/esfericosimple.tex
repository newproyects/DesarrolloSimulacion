\documentclass[10pt,a4papper]{article}
\usepackage{graphicx}
\usepackage{amsmath}
\usepackage{amssymb}
\usepackage{cancel}
\usepackage{multicol}
\usepackage{blindtext}
\usepackage[hidelinks]{hyperref}
\usepackage[left=2.00cm, right=3.00cm, top=2.00cm, bottom=2.00cm]{geometry}
\author{Angel Fdo. García Núñez}
\date{Septiembre 17, 2020}
\title{Estadisitica}

\begin{document}

\Huge
Desarrollo de péndulo esférico\\

Angel Fernando García Núñez

\newpage
\LARGE
Formulación de Euler-Lagrange\\\\
\Large
\[\frac{d}{dt}\left[\frac{\partial\mathcal{L}}{\partial\dot q}\right]-\frac{\partial\mathcal{L}}{\partial q}=0\quad,\quad\mathcal{L}=T-V\]\\

Coordenadas generalizadas
\[q_1=\theta\]
\[q_2=\varphi\]

\[x=l\cos\theta\sin\varphi\]
\[y=l\sin\theta\sin\varphi\]
\[z=-l\cos\varphi\]

\[\dot x=-l\dot\theta\sin\theta\sin\varphi+l\dot\varphi\cos\theta\cos\varphi\]
\[\dot y=l\dot\theta\cos\theta\sin\varphi+l\dot\varphi\sin\theta\cos\varphi\]
\[\dot z=l\dot\varphi\sin\varphi\]

\[\dot x^2=l^2\dot\theta^2\sin^2\theta\sin^2\varphi+l^2\dot\varphi^2\cos^2\theta\cos^2\varphi-2l^2\dot\theta\dot\varphi\cos\theta\sin\theta\cos\varphi\sin\varphi\]
\[\dot y^2=l^2\dot\theta^2\cos^2\theta\sin^2\varphi+l^2\dot\varphi^2\sin^2\theta\cos^2\varphi+2l^2\dot\theta\dot\varphi\cos\theta\sin\theta\cos\varphi\sin\varphi\]
\[\dot z^2=l^2\dot\varphi^2\sin^2\varphi\]

\[\dot x^2+\dot y^2+\dot z^2=l^2\dot\theta^2\sin^2\varphi+l^2\dot\varphi^2\]

\newpage
Energía potencial

\[V=mgz\to V=-mgl\cos\varphi\]

Energía cinética

\[T=\frac{1}{2}m(\dot x^2+\dot y^2+\dot z^2)\]
\[T=\frac{1}{2}m(l^2\dot\theta^2\sin^2\varphi+l^2\dot\varphi^2)\]\\
  
Lagrangiano

\[\mathcal{L}=T-V\]

\[\mathcal{L}=\frac{1}{2}m(l^2\dot\theta^2\sin^2\varphi+l^2\dot\varphi^2)+mgl\cos\varphi\]\\

Derivadas parciales del Lagrangiano

\[\frac{\partial\mathcal{L}}{\partial\theta}=0\]
\[\frac{\partial\mathcal{L}}{\partial\dot\theta}=ml^2\dot\theta\sin^2\varphi\]
\[\frac{d}{dt}\left[\frac{\partial\mathcal{L}}{\partial\dot\theta}\right]=ml^2\ddot\theta\sin^2\varphi+ml^2\dot\theta\dot\varphi\sin2\varphi\]\\

\[\frac{\partial\mathcal{L}}{\partial\varphi}=ml^2\dot\theta^2\sin\varphi\cos\varphi-mgl\sin\varphi\]
\[\frac{\partial\mathcal{L}}{\partial\dot\varphi}=ml^2\dot\varphi\]
\[\frac{d}{dt}\left[\frac{\partial\mathcal{L}}{\partial\dot\varphi}\right]=ml^2\ddot\varphi\]

\newpage
Ecuación de movimiento

\[\frac{d}{dt}\left[\frac{\partial\mathcal{L}}{\partial\dot\theta}\right]-\frac{\partial\mathcal{L}}{\partial\theta}=0\]\\

\[\frac{d}{dt}\left[ml^2\dot\theta\sin^2\varphi\right]=0\quad\to\quad\frac{d}{dt}\left[\dot\theta\sin^2\varphi\right]=0\]

\[\boxed{h=\dot\theta\sin^2\varphi\quad\to\quad h\equiv\text{cte.}}\]\\

\[ml^2\ddot\theta\sin^2\varphi+ml^2\dot\theta\dot\varphi\sin2\varphi=0\quad\to\quad\ddot\theta\sin^2\varphi+2\dot\theta\dot\varphi\cos\varphi\sin\varphi=0\]

\[\boxed{\ddot\theta\sin\varphi+2\dot\theta\dot\varphi\cos\varphi=0}\]

\[\ddot\theta\sin^3\varphi+2\dot\theta\dot\varphi\cos\varphi\sin^2\varphi=0\quad\to\quad\ddot\theta\sin^3\varphi+2h\dot\varphi\cos\varphi=0\]

\[\boxed{\ddot\theta(\varphi,\dot\varphi)=-2h\dot\varphi\frac{\cos\varphi}{\sin^3\varphi}}\]

\newpage
\[\frac{d}{dt}\left[\frac{\partial\mathcal{L}}{\partial\dot\varphi}\right]-\frac{\partial\mathcal{L}}{\partial\varphi}=0\]\\

\[ml^2\ddot\varphi-ml^2\dot\theta^2\sin\varphi\cos\varphi+mgl\sin\varphi=0\]

\[\ddot\varphi-\dot\theta^2\sin\varphi\cos\varphi+\frac{g}{l}\sin\varphi=0\quad\to\quad
\ddot\varphi-h^2\frac{\cos\varphi}{\sin^3\varphi}+\frac{g}{l}\sin\varphi=0\]

\[\omega_0^2=\frac{g}{l}\quad\to\quad\ddot\varphi-h^2\frac{\cos\varphi}{\sin^3\varphi}+\omega_0^2\sin\varphi=0\quad\to\quad
\frac{d}{dt}\left[\frac{1}{2}\dot\varphi^2+\frac{h^2}{2\sin^2\varphi}-\omega_0^2\cos\varphi\right]=0\]

\[\boxed{\varepsilon=\frac{1}{2}\dot\varphi^2+\frac{h^2}{2\sin^2\varphi}-\omega_0^2\cos\varphi\quad\to\quad\varepsilon\equiv\text{cte.}}\]\\

\[\boxed{\ddot\varphi(\varphi)=h^2\frac{\cos\varphi}{\sin^3\varphi}-\omega_0^2\sin\varphi}\]

\newpage
Solución númerica por método de Euler

\[f_{i+1}=f_i+h\dot f(f_i,t_i)\]\\

\[h=\dot\theta\sin^2\varphi\quad\to\quad\dot\theta(\varphi)=\frac{h}{\sin^2\varphi}\]

\[\varepsilon=\frac{1}{2}\dot\varphi^2+\frac{h^2}{2\sin^2\varphi}-\omega_0^2\cos\varphi\quad\to\quad
\dot\varphi(\varphi)=\sqrt{2\varepsilon-\frac{h^2}{\sin^2\varphi}+2\omega_0^2\cos\varphi}\]\\

\[\boxed{
  \begin{array}{rcl}
    t_{i+1}=t_i+h\\
    \theta_{i+1}=\theta_i+h\dot\theta(\varphi_{i})\\
    \varphi_{i+1}=\varphi_i+h\dot\varphi(\varphi_{i})
  \end{array}
}
\]

\newpage
Solución númerica por método de Euler

\[f_{i+1}=f_i+h\dot f(f_i,t_i)\]\\

\[\ddot\theta(\varphi,\dot\varphi)=-2h\dot\varphi\frac{\cos\varphi}{\sin^3\varphi}\]

\[\ddot\varphi(\varphi)=h^2\frac{\cos\varphi}{\sin^3\varphi}-\omega_0^2\sin\varphi\]

\[\boxed{
  \begin{array}{rcl}
    t_{i+1}=t_i+h\\\\
    \theta_{i+1}=\theta_i+h\dot\theta_{i}\\
    \dot\theta_{i+1}=\dot\theta_i+h\ddot\theta(\varphi_i,\dot\varphi_i)\\\\
    \varphi_{i+1}=\varphi_i+h\dot\varphi_{i}\\
    \dot\varphi_{i+1}=\dot\varphi_i+h\ddot\varphi(\varphi_i)
  \end{array}
}
\]

\end{document}
