\documentclass[10pt,a4papper]{article}
\usepackage{graphicx}
\usepackage{amsmath}
\usepackage{amssymb}
\usepackage{cancel}
\usepackage[hidelinks]{hyperref}
\usepackage[left=1.00cm, right=3.00cm, top=2.00cm, bottom=2.00cm]{geometry}
\author{Angel Fdo. García Núñez}
\date{Septiembre 17, 2020}
\title{Estadisitica}

\begin{document}

\Huge
Desarrollo y simulación de péndulo de masa\\

Angel Fernando García Núñez

\newpage
\LARGE
Formulación de Newton\\\\
\Large
\[\sum\vec\tau=I\vec\alpha\]
\[\sum\vec\tau=\sum\vec r\times\vec F\]\\
Sumatoria de torcas
\[\vec r\equiv\text{vector pivote - centro de masa}\]
\[\vec W\equiv\text{peso}\]
\[\sum\vec\tau=\vec r\times\vec W\]\\
\[\therefore\quad I\vec\alpha=\vec r\times\vec W\]
Forma escalar
\[|\vec r|=l\quad,\quad l\equiv\text{cte}\]
\[I\alpha=Wl\sin\theta\]\\
Ecuación de movimiento\\
\[W=-mg\]
\[\alpha=\frac{d^2\theta}{dt^2}\]\\
\[\boxed{\therefore\quad\frac{d^2\theta}{dt^2}+\frac{m}{I}gl\sin\theta=0}\]

\newpage
\LARGE
Formulación de Euler-Lagrange\\\\
\Large
\[\frac{d}{dt}\left[\frac{\partial\mathcal{L}}{\partial\dot q}\right]-\frac{\partial\mathcal{L}}{\partial q}=0\quad,\quad\mathcal{L}=T-V\]\\
Coordenadas generalizadas
\[q_1=\theta_1\]\\
\[l=\frac{L}{2}\]
\[x=l\sin\theta_1\quad,\quad\dot x=l\dot\theta_1\cos\theta_1\]
\[y=-l\cos\theta_1\quad,\quad\dot y=l\dot\theta_1\sin\theta_1\]\\
Energía potencial
\[V=mgy\to V=-mgl\cos\theta_1\]\\
Energía cinética
\[T=\frac{1}{2}\dot{\vec\theta}\cdot\vec L\to T=\frac{1}{2}\dot{\vec\theta}\cdot I\dot{\vec\theta}\to T=\frac{1}{2}I\dot\theta^2\]
\[T=\frac{1}{2}(I_1\dot\theta_1^2+I_2\dot\theta_2^2+I_3\dot\theta_3^2)\to T=\frac{1}{2}(I_1\dot\theta_1^2+I_2\cancelto{0}{\dot\theta_2^2}+I_3\cancelto{0}{\dot\theta_3^2})\]\\
\[T=\frac{1}{2}I_1\dot\theta_1^2\]\\
\[\therefore\quad\theta_1=\theta\quad,\quad I_1=I\to V=-mgl\cos\theta\quad,\quad T=\frac{1}{2}I\dot\theta^2\]

\newpage
Lagrangiano
\[\mathcal{L}=T-V\]
\[\mathcal{L}=\frac{1}{2}I\dot\theta^2+mgl\cos\theta\]\\
Derivadas del Lagrangiano
\[\frac{\partial\mathcal{L}}{\partial\theta}=-mgl\sin\theta\]
\[\frac{\partial\mathcal{L}}{\partial\dot\theta}=I\dot\theta\]
\[\frac{d}{dt}\left[\frac{\partial\mathcal{L}}{\partial\dot\theta}\right]=I\ddot\theta\]\\\\
Ecuación de movimiento
\[\frac{d}{dt}\left[\frac{\partial\mathcal{L}}{\partial\dot\theta}\right]-\frac{\partial\mathcal{L}}{\partial\theta}=0\]
\[I\ddot\theta+mgl\sin\theta=0\]\\
\[\boxed{\therefore\quad\ddot\theta+\frac{m}{I}gl\sin\theta=0}\]

\newpage
Solución analítica aproximada para ángulos pequeños
\[\theta\ll\frac{\pi}{2}\to\sin\theta\approx\theta\]
\[\therefore\quad\frac{m}{I}gl\sin\theta\approx\frac{m}{I}gl\theta\]\\
\[\ddot\theta+\frac{m}{I}gl\sin\theta=0\to\quad\ddot\theta+\frac{m}{I}gl\theta=0\]\\
Ecuación no lineal de segundo orden
\[\ddot\theta+\frac{m}{I}gl\theta=0\]
\[\omega=\sqrt{\frac{m}{I}gl}\]\\
\[\theta=e^{\beta t}\]
\[\dot\theta=\beta e^{\beta t}\]
\[\ddot\theta=\beta^2e^{\beta t}\]\\
\[\ddot\theta+\frac{m}{I}gl\theta=0\to\beta^2e^{\beta t}+\omega^2e^{\beta t}=0\]
\[\beta^2=-\omega^2\to\beta=\sqrt{-\omega^2}\to\beta=\omega\sqrt{-1}\]
\[\beta=\pm i\omega\]

\newpage
\[\theta=C_1e^{i\omega t}+C_2e^{-i\omega t}\]
\[\theta=C_1e^{i\omega t}+C_2e^{-i\omega t}\]
\[e^{\pm ik}=\cos k\pm i\sin k\]
\[\theta=C_1\cos(\omega t)+iC_1\sin(\omega t)+C_2\cos(\omega t)-iC_2\sin(\omega t)\]
\[\theta=(C_1+C_2)\cos(\omega t)+(C_1-C_2)i\sin(\omega t)\]
\[A=C_1+C_2\quad,\quad B=i(C_1-C_2)\]\\
\[\theta=A\cos(\omega t)+B\sin(\omega t)\]
\[\dot\theta=-A\omega\sin(\omega t)+B\omega\cos(\omega t)\]\\
\[\theta(t=0)=\theta_0\quad,\quad\theta(t=0)=A\cancelto{1}{\cos(\omega(0))}+B\cancelto{0}{\sin(\omega(0))}\to A=\theta_0\]
\[\dot\theta(t=0)=0\quad,\quad\dot\theta(t=0)=-A\cancelto{0}{\omega\sin(\omega(0))}+B\cancelto{1}{\omega\cos(\omega(0))}\to B=0\]\\
\[\therefore\quad\theta=Acos(\phi+\omega t)\]\\
\[\boxed{\theta(t)=Acos\left(\phi+t\sqrt{\frac{m}{I}gl}\right)}\]

\newpage
Solución númerica por método de Euler
\[x_{i+1}=x_i+h\dot x(x_i,t_i)\]\\
\[\ddot\theta+\frac{m}{I}gl\sin\theta=0\]
\[\omega=\dot\theta\to\dot\omega=\ddot\theta\]
\[\dot\omega=-\frac{m}{I}gl\sin\theta\]\\
\[\boxed{
  \begin{array}{rcl}
    t_{i+1}=t_i+h\\
    \theta_{i+1}=\theta_i+h\omega_{i+1}\\
    \omega_{i+1}=\omega_i-h\frac{m}{I}gl\sin\theta_i
  \end{array}
}\]\\

\newpage
\LARGE
Tensor de inercia\\\\
\Large
\[I=\begin{bmatrix}
I_{11} & I_{12} & I_{13}\\
I_{21} & I_{22} & I_{23}\\
I_{31} & I_{32} & I_{33}
\end{bmatrix}\]\\
\[I_{ij}=I_{ji}=\int_M\left[\delta_{ij}r^2-x_ix_j\right]dm=\int_V\rho\left[\delta_{ij}r^2-x_ix_j\right]dV\]\\\\
Momentos principales\\
\[I_i=I_{ii}\]
\[I_1=\int_V\rho(x_2^2+x_3^2)dV\]
\[I_2=\int_V\rho(x_1^2+x_3^2)dV\]
\[I_3=\int_V\rho(x_1^2+x_2^2)dV\]

\newpage
\LARGE
Barra de densidad constante\\\\
\Large
Momento de inercia en barra unidimensional\\
\[I=\lambda\int_0^Lr^2dr\to I=\frac{1}{3}\lambda|r^3|_0^L\]
\[I=\frac{1}{3}\lambda L^3\]
\[m=\lambda L\]\\
\[\boxed{\therefore\quad I=\frac{1}{3}mL^2}\]

\newpage
\LARGE
Cilindro de densidad constante\\\\
\Large
Parametrización con coordenadas polares
\[x_1=r\cos\theta\]
\[x_2=r\sin\theta\]
\[x_3=x_3\]
\[I_1=\rho\int_V(x_2(r,\theta)^2+x_3^2)|J(r,\theta,x_3)|dV\]
\[I_2=\rho\int_V(x_1(r,\theta)^2+x_3^2)|J(r,\theta,x_3)|dV\]
\[I_3=\rho\int_V(x_1(r,\theta)^2+x_2(r,\theta)^2)|J(r,\theta,x_3)|dV\]\\
Jacobiano\\
\[|J(r,\theta,x_3)|=\begin{vmatrix}\\
\frac{\partial x_1}{\partial r} & \frac{\partial x_1}{\partial\theta} & \frac{\partial x_1}{\partial x_3}\\
\frac{\partial x_2}{\partial r} & \frac{\partial x_2}{\partial\theta} & \frac{\partial x_2}{\partial x_3}\\
\frac{\partial x_3}{\partial r} & \frac{\partial x_3}{\partial\theta} & \frac{\partial x_3}{\partial x_3}
\end{vmatrix}\to
\[|J(r,\theta,x_3)|=
\begin{vmatrix}
  \cos\theta & -r\sin\theta & 0\\
  \sin\theta & r\cos\theta & 0\\
  0 & 0 & 1
\end{vmatrix}\]\\
\[|J(r,\theta,x_3)|=r\]\\

\newpage
Momento de inercia eje 1\\
\[I_1=\rho\int_V(x_2(r,\theta)^2+x_3^2)|J(r,\theta,x_3)|dV\to I_1=\rho\int_V(r^2\sin^2\theta+x_3^2)rdV\]\\
\[I_1=\rho\left[\int_Vr^3\sin^2\theta dV+\int_Vx_3^2rdV\right]\]
\[I_1=\rho\left[\int_0^Rr^3\int_0^{2\pi}\sin^2\theta\int_0^L dx_3d\theta dr+\int_0^Lx_3^2\int_0^Rr\int_0^{2\pi}d\theta drdx_3\right]\]
\large
\[\sin^2\theta=\frac{1-\cos2\theta}{2}\to I_1=\rho\left[\int_0^Rr^3\left(\frac{1}{2}\int_0^{2\pi}d\theta-\frac{1}{2}\int_0^{2\pi}\cos2\theta d\theta\right)\int_0^L dx_3d\theta dr+\int_0^Lx_3^2\int_0^Rr\int_0^{2\pi}d\theta drdx_3\right]\]\\
\Large
\[I_1=\rho\left[\frac{1}{4}\left|r^4\right|_0^R\left(\frac{1}{2}\left|\theta\right|_0^{2\pi}-\frac{1}{4}\cancelto{0}{\left|\sin2\theta\right|_0^{2\pi}}\right)\left|x_3\right|_0^L+\frac{1}{3}\left|x_3^3\right|_0^L\frac{1}{2}\left|r^2\right|_0^R\left|\theta\right|_0^{2\pi}\right]\to I_1=\frac{1}{4}\rho\pi LR^4+\frac{1}{3}\rho\pi L^3R^2\]


Momento de inercia eje 2\\
\[I_2=\rho\int_V(x_1(r,\theta)^2+x_3^2)|J(r,\theta,x_3)|dV\to I_2=\rho\int_V(r^2\cos^2\theta+x_3^2)rdV\]\\
\[I_2=\rho\left[\int_Vr^3\cos^2\theta dV+\int_Vx_3^2rdV\right]\]
\[I_2=\rho\left[\int_0^Rr^3\int_0^{2\pi}\cos^2\theta\int_0^L dx_3d\theta dr+\int_0^Lx_3^2\int_0^Rr\int_0^{2\pi}d\theta drdx_3\right]\]
\large
\[\cos^2\theta=\frac{1+\cos2\theta}{2}\to I_2=\rho\left[\int_0^Rr^3\left(\frac{1}{2}\int_0^{2\pi}d\theta+\frac{1}{2}\int_0^{2\pi}\cos2\theta d\theta\right)\int_0^L dx_3d\theta dr+\int_0^Lx_3^2\int_0^Rr\int_0^{2\pi}d\theta drdx_3\right]\]\\
\Large
\[I_2=\rho\left[\frac{1}{4}\left|r^4\right|_0^R\left(\frac{1}{2}\left|\theta\right|_0^{2\pi}+\frac{1}{4}\cancelto{0}{\left|\sin2\theta\right|_0^{2\pi}}\right)\left|x_3\right|_0^L+\frac{1}{3}\left|x_3^3\right|_0^L\frac{1}{2}\left|r^2\right|_0^R\left|\theta\right|_0^{2\pi}\right]\to I_2=\frac{1}{4}\rho\pi LR^4+\frac{1}{3}\rho\pi L^3R^2\]

\newpage
Momento de inercia eje 3\\
\[I_3=\rho\int_V(x_1(r,\theta)^2+x_2(r,\theta)^2)|J(r,\theta,x_3)|dV\to I_3=\rho\int_Vr^3dV\]\\
\[I_3=\rho\int_0^Rr^3\int_0^{2\pi}\int_0^L dx_3d\theta dr\to I_3=\rho\frac{1}{4}\left|r^4\right|_0^R\left|\theta\right|_0^{2\pi}\left|x_3\right|_0^L\to I_3=\frac{1}{2}\rho\pi LR^4\]\\\\
Momentos principales\\
\[V=\pi LR^2\quad,\quad m=\rho V\]\\
\[\boxed{
  \begin{array}{rcl}
    I_1=\frac{1}{4}mR^2+\frac{1}{3}mL^2\\\\
    I_2=\frac{1}{4}mR^2+\frac{1}{3}mL^2\\\\
    I_3=\frac{1}{2}mR^2
  \end{array}
}\]

\end{document}
