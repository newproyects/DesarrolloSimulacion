\documentclass[10pt,a4papper]{article}
\usepackage{graphicx}
\usepackage{amsmath}
\usepackage{amssymb}
\usepackage{cancel}
\usepackage{multicol}
\usepackage{blindtext}
\usepackage{tikz}
\usepackage[hidelinks]{hyperref}
\usepackage[left=2.00cm, right=3.00cm, top=2.00cm, bottom=2.00cm]{geometry}
\author{Angel Fdo. García Núñez}
\date{Enero 18, 2023}
\title{Estadisitica}

\begin{document}

\Huge
Compuertas OR-AND usando perceptron\\

Angel Fernando García Núñez

\Large
\newpage
\[\text{Compuerta OR:}\]
\[\begin{tabular}{| c | c | c |}
  \hline
  $x_1$ & $x_2$ & $y$\\
  \hline
  0 & 0 & 0\\
  0 & 1 & 1\\
  1 & 0 & 1\\
  1 & 1 & 1\\
  \hline
\end{tabular}\]

\[\text{Compuerta AND:}\]
\[\begin{tabular}{| c | c | c |}
  \hline
  $x_1$ & $x_2$ & $y$\\
  \hline
  0 & 0 & 0\\
  0 & 1 & 0\\
  1 & 0 & 0\\
  1 & 1 & 1\\
  \hline
\end{tabular}\]\\

\[\text{Perceptron}\]

\begin{center}
  \begin{tikzpicture}
    \node[draw=none](x1)at(0,0){$x_1$};
    \node[draw=none](x2)at(0,-2){$x_2$};
    \node[draw,shape=circle](w1)at(2,0){$w_1$};
    \node[draw,shape=circle](w2)at(2,-2){$w_2$};
    \node[draw=none](b)at(4,1){$b$};
    \node[draw,shape=circle](n)at(4,-1){$\sum$};
    \node[draw](act)at(6,-1){$\varphi$};
    \node[draw=none](y)at(8,-1){$y$};
    
    \draw[->](x1)to(w1);
    \draw[->](x2)to(w2);
    \draw[->](b)to(n);
    \draw[->](w1)to(n);
    \draw[->](w2)to(n);
    \draw[->](n)to(act);
    \draw[->](act)to(y);
  \end{tikzpicture}
\end{center}

\[b=w_0x_0\quad:\quad x_0=1\]

\begin{center}
  \begin{tikzpicture}
    \node[draw=none](x0)at(0,0){$x_0=1$};
    \node[draw=none](x1)at(0,-2){$x_1$};
    \node[draw=none](x2)at(0,-4){$x_2$};
    \node[draw,shape=circle](w0)at(2,0){$w_0$};
    \node[draw,shape=circle](w1)at(2,-2){$w_1$};
    \node[draw,shape=circle](w2)at(2,-4){$w_2$};
    \node[draw,shape=circle](n)at(4,-2){$\sum$};
    \node[draw](act)at(6,-2){$\varphi$};
    \node[draw=none](y)at(8,-2){$y$};
    
    \draw[->](x0)to(w0);
    \draw[->](x1)to(w1);
    \draw[->](x2)to(w2);
    \draw[->](w0)to(n);
    \draw[->](w1)to(n);
    \draw[->](w2)to(n);
    \draw[->](n)to(act);
    \draw[->](act)to(y);
  \end{tikzpicture}
\end{center}

\newpage
\begin{center}
  \begin{tikzpicture}
    \node[draw,shape=circle](x0)at(0,0){$x_0$};
    \node[draw,shape=circle](x1)at(0,-2){$x_1$};
    \node[draw,shape=circle](x2)at(0,-4){$x_2$};
    \node[draw,shape=circle](n)at(3,-2);
    \node[draw,shape=circle](y)at(5,-2){$y$};
    
    \draw[->](x0)to node[draw=none,above]{$w_0$}(n);
    \draw[->](x1)to node[draw=none,above]{$w_1$}(n);
    \draw[->](x2)to node[draw=none,above]{$w_2$}(n);
    \draw[->](n)to node[draw=none,above]{$\varphi$}(y);
  \end{tikzpicture}
\end{center}

\[\text{Comportamiento del perceptron}\]

\[\vec x=(1,x_1,x_2)\quad,\quad\vec w=(w_0,w_1,w_2)\]

\[\varphi(s)=
\left\{\begin{array}{ccc}
1 & : & s\geq 0 \\
0 & : & s<0
\end{array}\]

\[y=\varphi(\vec x\cdot\vec w)\]\\

\[\text{Aprendizaje}\]

\[\text{Vector de pesos en la iteración }i\]
\[\vec w_i=(w_{i,0},w_{i,1},w_{i,2})\]

\[\text{Función de error}\]
\[e=y-y_i\]

\[\text{Factor de aprendizaje}\]
\[\alpha\in(0,1)\]

\[\text{Algoritmo de aprendizaje}\]
\[w_{i+1,j}=w_{i,j}+\alpha ex_j\]

\newpage

\newpage

\newpage

\newpage

\newpage

\newpage

\newpage

\newpage

\newpage

\newpage

\newpage

\newpage

\newpage

\newpage

\newpage

\newpage

\newpage

\newpage

\newpage

\newpage

\newpage

\newpage

\newpage


\end{document}
